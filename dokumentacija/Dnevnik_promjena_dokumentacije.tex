\chapter{Dnevnik promjena dokumentacije}
		
		\textbf{\textit{Kontinuirano osvježavanje}}\\
				
		
		\begin{longtabu} to \textwidth {|X[2, l]|X[13, l]|X[3, l]|X[3, l]|}
			\hline \multicolumn{1}{|l|}{\textbf{Rev.}}	& \multicolumn{1}{l|}{\textbf{Opis promjene/dodatka}} & \multicolumn{1}{|l|}{\textbf{Autori}} & \multicolumn{1}{l|}{\textbf{Datum}} \\[3pt] \hline
			\endfirsthead
			
			\hline \multicolumn{1}{|l|}{\textbf{Rev.}}	& \multicolumn{1}{l|}{\textbf{Opis promjene/dodatka}} & \multicolumn{1}{|l|}{\textbf{Autori}} & \multicolumn{1}{l|}{\textbf{Datum}} \\[3pt] \hline
			\endhead
			
			\hline 
			\endlastfoot
			
			0.1 & Napravljen predložak.	& Roček & 14.10.2020. 		\\[3pt] \hline 
			0.2	& Opis projektnog zadatka.  &  Oreč  & 15.10.2020.	\\[3pt] \hline 
			0.2.1 & Započeta razrada UC dijagrama. & Bokšić \ Roček \ Ćurić \ Đaković & 15.10.2020.	\\[3pt] \hline 
			0.2.2 & Izmjena predloška dokumentacije.  &  Bokšić  & 15.10.2020.	\\[3pt] \hline
			0.2.3 & Opis obrazaca uporabe  &  Lipovac  & 15.10.2020.	\\[3pt] \hline
			0.2.4 & Funkcionalni zahtjevi  &  Roček \ Jakas  & 15.10.2020.	\\[3pt] \hline
			0.3 & Dijagram obrazaca uporabe & Roček & 20.10.2020.\\[3pt] \hline
			0.4 & Ispravak opisa obrazaca uporabe. & Lipovac \newline Bokšić & 22.10.2020. \\[3pt] \hline
			0.4.1 &  Unesen dnevnik sastajanja.  & Bokšić & 22.10.2020. \\[3pt] \hline
			0.4.2 & Izmjene na opisu i funkcijskim zahtjevima & Oreč & 22.10.2020.   \\[3pt] \hline
			0.5 & Sekvencijski dijagram za registraciju. & Đaković & 23.10.2020.  \\[3pt] \hline
			0.5.1 & Sekvencijski dijagram za zadavanje zahtjeva. & Jakas & 24.10.2020.  \\[3pt] \hline
			0.6 & Izmjena UC dijagrama & Oreč & 30.10.2020.  \\[3pt] \hline
			0.7 & Svi sekvencijski dijagrami & Ćurić & 30.10.2020.  \\[3pt] \hline
			0.8 & Izmjena opisa obrazaca uporabe & Lipovac & 30.10.2020. \\[3pt] \hline
			0.8.1 & Opisi sekvencijskih dijagrama  & Bokšić & 03.11.2020. \\[3pt] \hline
			0.8.2 &  Dodane tablice za bazu. \newline Dodana slika baze. & Bokšić & 03.11.2020. \\[3pt] \hline
			0.8.3 &  Unos dnevnika sastajanja.& Bokšić & 03.11.2020. \\[3pt] \hline
			0.9 & Izmjena opisa zadatka i dodani ostali zahtjevi & Oreč & 04.11.2020. \\[3pt] \hline 
			0.9.1 & Opis tablica baze podataka & Lipovac \newline Ćurić & 04.11.2020. \\[3pt] \hline  
			0.9.2 & Opis arhitekture sustava & Bokšić & 05.11.2020. \\[3pt] \hline
			0.9.3 & Izmijenjena baza podataka i tablice entiteta & Bokšić & 11.11.2020. \\[3pt] \hline
			0.10 & Završeni dijagrami razreda i opisi istih & Oreč \newline Lipovac   &  11.11.2020  \\[3pt] \hline
			\textbf{1.0} & Verzija samo s bitnim dijelovima za 1. ciklus & Bokšić & 12.11.2020. \\[3pt] 	\hline 
		\end{longtabu}
		
		\begin{comment}
			0.12.1 & Započeo dijagrame razreda & Horvat & 10.09.2013. \\[3pt] \hline 
			0.12.2 & Nastavak dijagrama razreda & Horvat & 11.09.2013. \\[3pt] \hline 
			\textbf
			{1.0} & Verzija samo s bitnim dijelovima za 1. ciklus & Ivošević & 11.09.2013. \\[3pt] 	\hline 
			1.1 & Uređivanje teksta -- funkcionalni i nefunkcionalni zahtjevi & Grudenić \newline Jović 		& 14.09.2013. \\[3pt] \hline 
			1.2 & Manje izmjene:Timer - Brojilo vremena & Grudenić & 15.09.2013. \\[3pt] \hline 
			1.3 & Popravljeni dijagrami obrazaca uporabe & Jović & 15.09.2013. \\[3pt] \hline 
			1.5 & Generalna revizija strukture dokumenta & Ivošević & 19.09.2013. \\[3pt] \hline 
			1.5.1 & Manja revizija (dijagram razmještaja) & Jović & 20.09.2013. \\[3pt] \hline 
			\textbf{2.0} & Konačni tekst predloška dokumentacije  & Ivošević & 28.09.2013. \\[3pt] 	\hline 
			&  &  & \\[3pt] \hline
		\end{comment}
		
		
		
		\newcommand{\comm}[3]{
		\textit{Moraju postojati glavne revizije dokumenata 1.0 i 2.0 na kraju prvog i drugog ciklusa. Između tih revizija mogu postojati manje revizije već prema tome kako se dokument bude nadopunjavao. Očekuje se da nakon svake značajnije promjene (dodatka, izmjene, uklanjanja dijelova teksta i popratnih grafičkih sadržaja) dokumenta se to zabilježi kao revizija. Npr., revizije unutar prvog ciklusa će imati oznake 0.1, 0.2, …, 0.9, 0.10, 0.11.. sve do konačne revizije prvog ciklusa 1.0. U drugom ciklusu se nastavlja s revizijama 1.1, 1.2, itd.}
		}