\chapter{Zaključak i budući rad}
\textit
Naš projektni zadatak predmeta "Programsko inženjerstvo" bio je napraviti web aplikaciju "Pomozi mi" koja bi omogućila korisnicima međusobno pružanje pomoći. Pomoć se odnosila na svakodnevne obveze ili poteškoće. Korisnici u svakom trenutku imaju mogućnost pregleda aktivnih zahtjeva i njihovog izvršavanja. Osim toga, mogu sami zatražiti pomoć ispunjavanjem obrasca u kojemu ga pobliže opisuju zajedno sa svim nužnim informacijama.\newline Naš prvi susret sastojao se od međusobog upoznavanja, komentiranja dane teme te organiziranja buduće komunikacije. Za raspodjelu poslova često smo koristili Gitlab, ali većina komunikacije i sastanaka odvijala se putem MS Teams i WhatsApp-a. U prvom ciklusu našega rada uglavnom smo se usredotočili na pisanje dokumentacije i upoznavanje sa novim tehnologijama. Svima nam je to bio prvi susret sa složenijim projektom, pa samim time i dokumentacijom, ali smo uz pomoć predavanja na kolegiju i kontinuiranu komunikaciju sa asistentima to uspješno savladali. Time smo dobili uvid u važnost dokumetiranja projekta unaprijed te kako komponente poput obrazaca uporabe ili različitih dijagrama pružaju pregled funkcionalnosti. Sve navedeno nam je poslužilo kao okvir za budući rad i uvelike olakšalo implementaciju.\newline
U drugoj fazi naglasak je bio na izradi same aplikacije. Neki članovi su se susreli sa alatima koje smo koristili za programsku implementaciju, no i oni su naišli na puno novih prepreka te proširili svoje znanje. U ovom razdoblju organizacija je bila ključna te se svaki član trudio držati dogovorenih rokova. Nakon napisanog koda uslijedilo je testiranje sustava, nadopunjavanje dokumentacije i završni prepravci. Međusobno smo si pomagali, što je rezultiralo ugodnom i produktivnijom atmosferom.\newline
Nakon 8 tjedana rada, aplikacija je bila završena. Iako je malo drugačija od početne vizije, tražene funkcionalnosti su uspješno implementirane. Jedna od svari koje smo naučili je da se prvobitni plan često mijenja i prilagođava tijekom rada na samom projektu. Neki od problema s kojima smo se susretali su bili rad s Gitom, uključivanje karte u aplikaciju i povezivanje korisničkog sučelja sa poslužiteljem. Sve prepreke smo na kraju uspješno savladali i iz njih stekli znanje. Cijelo iskustvo dalo nam je uvid u timski rad i sve faze izrade projekta te će nam svima biti koristan za osobni rast kao budućih inženjera.