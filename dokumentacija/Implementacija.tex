\chapter{Implementacija i korisničko sučelje}
		
		\section{Korištene tehnologije i alati}
		
			%\textbf{\textit{dio 2. revizije}}\\
			
            Komunikacija u timu realizirana je korištenjem aplikacije\underline{ Discord}\footnote{\url{https://discord.com/}},\underline{ MicrosoftTeams}\footnote{\url{https://www.microsoft.com/en-us/microsoft-365/microsoft-teams}} i \underline{WhatsApp}\footnote{\url{https://www.whatsapp.com/}}. Za izradu UML dijagrama korišten je alat \underline{Astah Professional}\footnote{\url{https://astah.net/}}, a kao sustav za upravljanje izvornim kodom \underline{Git}\footnote{\url{https://git-scm.com/}}. Udaljeni repozitorij projekta je dostupan na web platformi \underline{GitLab}\footnote{\url{https://about.gitlab.com/}}.
            Kao razvojno okruženje korišteni su \underline{IntelliJ}
            \footnote{\url{https://www.jetbrains.com/idea/}} - integrirano razvojno okruženje (IDE) tvrtke JetBrains.
            \underline{Spring Tool Suite}
            \footnote{\url{https://spring.io/tools}} (Eclipse) - integrirano razvojno okruženje namijenjeno za razvoj aplikacija koje koriste Spring kao radni okvir te \underline{VSCode}\footnote{\url{https://code.visualstudio.com/}} - integrirano razvojno okruženje pogodno za razvoj frontend-a.
            
            Aplikacija je napisana koristeći  radni
            okvir \underline{Spring Boot}\footnote{\url{https://spring.io/projects/spring-boot}} i jezik \underline{Javu}\footnote{\url{https://www.java.com/}} za
            izradu \emph{backenda} te \underline{React}\footnote{\url{https://reactjs.org/}} i jezik \underline{JavaScript}\footnote{\url{https://www.javascript.com/}} za izradu \emph{frontenda}.
            Spring Boot je open-source Javin radni okvir koji omogućuje izgradnju mikro servisa. Sam Spring Boot dolazi s već predkonfiguriranim značajkama koje programerima omogućuju konvencijonalnost te mogućnost pokretanja aplikacije bez dodatnog posla. 
            React, također poznat kao React.js ili ReactJS, je biblioteka u JavaScriptu za izgradnju korisničkih
            sučelja. Složene aplikacije u React-u obično zahtijevaju korištenje dodatnih biblioteka za interakciju s API-jem.
            
            Baza podataka i aplikacija se nalaze na poslužitelju \underline{Heroku}
            \footnote{\url{https://www.heroku.com/}}.
            
        
\newpage 

\begin{comment}   
        \section{Ispitivanje programskog rješenja}
			
			\textbf{\textit{dio 2. revizije}}
			
			Svi testovi izvršeni su pomoću Junit i Selenium. Ispitivanje se radilo po obrascima uporabe kako bi
            se provjerila osnovna funkcionalnost sustava, ali i nasumičnim kretanjima po aplikaciji
            kako bi se pronašle neočekivane greške  ili nepredvidena ponašanja.
            Svaki dio sustava je ispitan, no zbog jednostavnosti u dokumentaciji će biti prikazan
            samo dio ispitivanja. Prikazivanje ispitivanja UC?, UC?, UC?,UC?, UC? i UC?.
            
\end{comment}

		    \section{Ispitivanje programskog rješenja}
			
			\textbf{\textit{dio 2. revizije}}\\
			
			 \textit{U ovom poglavlju je potrebno opisati provedbu ispitivanja implementiranih funkcionalnosti na razini komponenti i na razini cijelog sustava s prikazom odabranih ispitnih slučajeva. Studenti trebaju ispitati temeljnu funkcionalnost i rubne uvjete.}
           
%\begin{comment}
        \subsection{Ispitivanje komponenti}
		    
		    \textbf{\textit{dio 2. revizije}}\\
    
			\textit{Potrebno je provesti ispitivanje jedinica (engl. unit testing) nad razredima koji implementiraju temeljne funkcionalnosti. Razraditi \textbf{minimalno 6 ispitnih slučajeva} u kojima će se ispitati redovni slučajevi, rubni uvjeti te izazivanje pogreške (engl. exception throwing). Poželjno je stvoriti i ispitni slučaj koji koristi funkcionalnosti koje nisu implementirane. Potrebno je priložiti izvorni kôd svih ispitnih slučajeva te prikaz rezultata izvođenja ispita u razvojnom okruženju (prolaz/pad ispita). }
			
			
			
			\subsection{Ispitivanje sustava}
			
			 \textit{Potrebno je provesti i opisati ispitivanje sustava koristeći radni okvir Selenium\footnote{\url{https://www.seleniumhq.org/}}. Razraditi \textbf{minimalno 4 ispitna slučaja} u kojima će se ispitati redovni slučajevi, rubni uvjeti te poziv funkcionalnosti koja nije implementirana/izaziva pogrešku kako bi se vidjelo na koji način sustav reagira kada nešto nije u potpunosti ostvareno. Ispitni slučaj se treba sastojati od ulaza (npr. korisničko ime i lozinka), očekivanog izlaza ili rezultata, koraka ispitivanja i dobivenog izlaza ili rezultata.\\ }
			 
			 \textit{Izradu ispitnih slučajeva pomoću radnog okvira Selenium moguće je provesti pomoću jednog od sljedeća dva alata:}
			 \begin{itemize}
			 	\item \textit{dodatak za preglednik \textbf{Selenium IDE} - snimanje korisnikovih akcija radi automatskog ponavljanja ispita	}
			 	\item \textit{\textbf{Selenium WebDriver} - podrška za pisanje ispita u jezicima Java, C\#, PHP koristeći posebno programsko sučelje.}
			 \end{itemize}
		 	\textit{Detalji o korištenju alata Selenium bit će prikazani na posebnom predavanju tijekom semestra.}
			
			\eject 
%\end{comment}
\section{Dijagram razmještaja}
	
	\textbf{\textit{dio 2. revizije}}
	
	\text Dijagrami razmještaja opisuju topologiju sklopovlja i programsku potporu koja se koristi u implementaciji sustava u njegovom radnom okruženju. Na poslužitelju
	(HerokuCloudServer) nalazi se sustav Ubunt koji ima postavljen Linux operacijski sustav te se na njemu nalaze HTTP Server i poslužitelj baze podataka . Klijenti koriste web preglednik na svom uređaju(PC ili mobitel) kako bi pristupili web aplikaciji.Sustav je baziran na arhitekturi ”klijent
    – poslužitelj”, a komunikacija izmedu računala korisnika (klijent, zaposlenik,vlasnik, administrator) i poslužitelja odvija se preko HTTP veze.  

    %unos slike
		\begin{figure}[H]
			\includegraphics[scale=0.5]{slike/Dijagram Razmještaja.png} %veličina slike u odnosu na originalnu datoteku i pozicija slike
			\centering
			\caption { Dijagram Razmještaja}
			\label{fig:5.1}
			\end{figure}


\begin{comment}	
	\textit{Potrebno je umetnuti \textbf{specifikacijski} dijagram razmještaja i opisati ga. Moguće je umjesto specifikacijskog dijagrama razmještaja umetnuti dijagram razmještaja instanci, pod uvjetom da taj dijagram bolje opisuje neki važniji dio sustava.}
	
	\eject 
\end{comment}

\newpage


\section{Upute za puštanje u pogon}

	\textbf{\textit{dio 2. revizije}}\\

	\textit{U ovom poglavlju potrebno je dati upute za puštanje u pogon (engl. deployment) ostvarene aplikacije. Na primjer, za web aplikacije, opisati postupak kojim se od izvornog kôda dolazi do potpuno postavljene baze podataka i poslužitelja koji odgovara na upite korisnika. Za mobilnu aplikaciju, postupak kojim se aplikacija izgradi, te postavi na neku od trgovina. Za stolnu (engl. desktop) aplikaciju, postupak kojim se aplikacija instalira na računalo. Ukoliko mobilne i stolne aplikacije komuniciraju s poslužiteljem i/ili bazom podataka, opisati i postupak njihovog postavljanja. Pri izradi uputa preporučuje se \textbf{naglasiti korake instalacije uporabom natuknica} te koristiti što je više moguće \textbf{slike ekrana} (engl. screenshots) kako bi upute bile jasne i jednostavne za slijediti.}
		
		
	\textit{Dovršenu aplikaciju potrebno je pokrenuti na javno dostupnom poslužitelju. Studentima se preporuča korištenje neke od sljedećih besplatnih usluga: \href{https://aws.amazon.com/}{Amazon AWS}, \href{https://azure.microsoft.com/en-us/}{Microsoft Azure} ili \href{https://www.heroku.com/}{Heroku}. Mobilne aplikacije trebaju biti objavljene na F-Droid, Google Play ili Amazon App trgovini.}
	
	
	\eject 

