\chapter{Specifikacija programske potpore}

\section{Funkcionalni zahtjevi}

\textbf{\textit{dio 1. revizije}}\\



\noindent \textbf{Dionici:}

\begin{packed_enum}
	
	\item Razvojni tim
	\item Administrator			
	\item Korsinik
	\item Vanjski suradnici - CROZ
	
\end{packed_enum}

\noindent \textbf{Aktori i njihovi funkcionalni zahtjevi:}


\begin{packed_enum}
	\item  \underbar{Korisnik (inicijator) može:}
	
	\begin{packed_enum}
		
		\item biti javni korisnik 
		\begin{packed_enum}
			
			\item   može se registrirati
			
		\end{packed_enum}
		\item biti registrirani korisnik 
		\begin{packed_enum}
			
			\item  prijavljuje se u sustav
			\item pregled liste korisnika
			\item pregled vlastitog profila
			\item zadavanje zahtjeva za pomoć
			\item prikaz liste aktivnih zahtjeva
		    \item razmjenjivati notifikacije kada izvršitelj odabere zahtjev
		    \item dohvatiti profil drugog korisnika u trenutku pregleda zahtjeva
		    \item dohvatiti profile ostalih korisnika iz liste korisnika
		    \item ocjenjivati i komentirati druge korisnike
		    \item vidjeti "lanac povjerenja"
		    \item pregledati listu svojih izvršenih i ponuđenih zahtjeva
		
		\end{packed_enum}
	\end{packed_enum}
	
	\eject
	
	\item  \underbar{Administrator (inicijator) može:}
	
	\begin{packed_enum}
		
		\item pregled liste korisnika
		\item prikaz liste aktivnih zahtjeva
		\item kontrolirati sadržaj koji se objavljuje
		\item brisanje zahtjeva
		\item blokiranje korisnika pristupu aplikaciji 
	\end{packed_enum}
	
	\item  \underbar{Baza podataka (sudionik) može:}
	\begin{packed_enum}
		
		\item spremanje podataka o profilima, zahtjevima, razmjeni notifikacija
		
	\end{packed_enum}
	
	\item  \underbar{Poslužitelj (sudionik) može:}
	\begin{packed_enum}
		
		\item spremanje podataka o profilima, zahtjevima, razmjeni notifikacija
		
	\end{packed_enum}
\end{packed_enum}

\eject 



\subsection{Obrasci uporabe}

\textbf{\textit{dio 1. revizije}}

\subsubsection{Opis obrazaca uporabe}
\textit{Funkcionalne zahtjeve razraditi u obliku obrazaca uporabe. Svaki obrazac je potrebno razraditi prema donjem predlošku. Ukoliko u nekom koraku može doći do odstupanja, potrebno je to odstupanje opisati i po mogućnosti ponuditi rješenje kojim bi se tijek obrasca vratio na osnovni tijek.}\\


\noindent \underbar{\textbf{UC1 -Prijava}}
\begin{packed_item}
	\item \textbf{Glavni sudionik: }korisnik
	\item  \textbf{Cilj:} dobiti pristup korisničkom sučelju
	\item  \textbf{Sudionici:} baza podataka, administrator
	\item  \textbf{Preduvjet:} registracija
	\item  \textbf{Opis osnovnog tijeka:}
	
	\item[] \begin{packed_enum}
		
		\item unos korisničkog imena i lozinke
		\item potvrda o ispravnosti unesenih podataka
		\item pristup korisničkim funkcijama
	\end{packed_enum}
	
	\item  \textbf{Opis mogućih odstupanja:}
	
	\item[] \begin{packed_item}
		
		\item[1.a] neispravan unos korisničkog imena ili lozinke
		\item[] \begin{packed_enum}
			
			\item sustav obaviještava korisnika o neispravnom unosu i ponovno ga vraća na stranicu za prijavu
			
		\end{packed_enum}
	\end{packed_item}
\end{packed_item}

\noindent \underbar{\textbf{UC2 -Pregled zahtjeva}}
\begin{packed_item}
	\item \textbf{Glavni sudionik: }korisnik
	\item  \textbf{Cilj:} pregledati aktualne i izvršene zahtjeve
	\item  \textbf{Sudionici:} baza podataka, administrator
	\item  \textbf{Preduvjet:} korisnik je prijavljen
	\item  \textbf{Opis osnovnog tijeka:}
	
	\item[] \begin{packed_enum}
		
		\item korisnik šalje zahtjev za čitanjem liste zahtjeva
		\item na njegovom ekranu se prikazuju svi aktivni zahtjevi koji se nalaze unutar određene geografske udaljenosti
		\item omogućeno sortiranje liste zahtjeva prema kategorijama
		\item korisniku je i omogućen pregled svih vlastitih zadanih i izvršenih zahtjeva
	\end{packed_enum}
	
	\item  \textbf{Opis mogućih odstupanja:}
	
	\item[] \begin{packed_item}
		
		\item[2.a] nema aktivnih zahtjeva
	\end{packed_item}
\end{packed_item}

\noindent \underbar{\textbf{UC3 -Upravljanje zahtjevima}}
\begin{packed_item}
	
	\item \textbf{Glavni sudionik: } autor (korisnik)
	\item  \textbf{Cilj:} stvaranje i uređivanje zahtjeva
	\item  \textbf{Sudionici:} baza podataka, administrator
	\item  \textbf{Preduvjet:} registracija
	\item  \textbf{Opis osnovnog tijeka:}
	
	\item[] \begin{packed_enum}
		
		\item korisnik odabire opciju za stvaranje novog zahtjeva ili njegovo uređivanje
		\item unos novih ili izmjenjenih informacija koje se zatim spremaju u bazu podataka nakon pritiska tipke "Spremi"
		\item baza podataka se ažurira
		\item mogućnost brisanja ili blokiranja vlastitog zahtjeva
	\end{packed_enum}
	
	\item  \textbf{Opis mogućih odstupanja:}
	
	\item[] \begin{packed_item}
		
		\item[2.a] korisnik unese podatke o zahtjevu, ali ne pritisne tipku "Spremi"
		\item[] \begin{packed_enum}
			
			\item sustav obaviještava korisnika da nije spremio podatke prije izlaska iz prozora
			
		\end{packed_enum}
		\item[5] administrator ima mogućnost brisanja neprimjerenih zahtjeva 
		
		
		
	\end{packed_item}
\end{packed_item}
\noindent \underbar{\textbf{UC4 - Dohvaćanje profila}}
\begin{packed_item}
	
	\item \textbf{Glavni sudionik: }korisnik
	\item  \textbf{Cilj:} dobiti pristup tuđim korisničkim profilima
	\item  \textbf{Sudionici:} baza podataka, administrator
	\item  \textbf{Preduvjet:} registracija
	\item  \textbf{Opis osnovnog tijeka:}
	
	\item[] \begin{packed_enum}
		
		\item dok korisnik donosi odluku o pomoći, otvara se opcija pregleda profila autora zahtjeva
		\item iz baze se dohvaća profil o autoru zahtjeva kao i "lanac povjerenja"
	\end{packed_enum}
\end{packed_item}
\noindent \underbar{\textbf{UC5 - Izvršavanje zahtjeva}}
\begin{packed_item}
	
	\item \textbf{Glavni sudionik: }izvršitelj zahtjeva (korisnik)
	\item  \textbf{Cilj:} izvršiti potvrđeni zahtjev
	\item  \textbf{Sudionici:} -
	\item  \textbf{Preduvjet:} registracija
	\item  \textbf{Opis osnovnog tijeka:}
	
	\item[] \begin{packed_enum}
		
		\item izvršitelj zahtjeva odabire zahtjev koji želi izvršiti
		\item omogućuje se razmjena notifikacija između izvršitelja i autora zahtjeva te se iz baze dohvaćaju sve potrebne informacije
		\item po izvršenju zahtjeva, izvršitelj označava u aplikaciji da je zahtjev izvršen
		\item omogućuje se međusobno ocjenjivanje korisnika ocjenama od 1 do 5, uz opcionalne komentare
		\item ocjene i osvrti se pohranjuju u bazu podataka i ažuriraju se prosječne ocjene korisnika
	\end{packed_enum}
	
	\item  \textbf{Opis mogućih odstupanja:}
	
	\item[] \begin{packed_item}
		
		\item[4.a] klijent ne želi napisati komentar
		\item[] \begin{packed_enum}
			
			\item sustav zatvara prozor za ocjenjivanje i pohranjuje se samo ocjena
			
		\end{packed_enum}
		
	\end{packed_item}
\end{packed_item}
\noindent \underbar{\textbf{UC6 - Dodjela administrativnog područja}}
\begin{packed_item}
	\item \textbf{Glavni sudionik: }administrator
	\item  \textbf{Cilj:} dodjela administrativnog područja prema geografskoj lokaciji
	\item  \textbf{Sudionici:} sustav
	\item  \textbf{Preduvjet:} korisnik je registriran i prijavljen kao administrator
	\item  \textbf{Opis osnovnog tijeka:}
	
	\item[] \begin{packed_enum}
		
		\item administratoru preko vlastite lokacije sustav dodjeljuje administrativno područje
	\end{packed_enum}
\end{packed_item}
\noindent \underbar{\textbf{UC7 - Registracija}}
\begin{packed_item}
	\item \textbf{Glavni sudionik: }korisnik
	\item  \textbf{Cilj:} stvoriti korisnički račun za pristup sustavu
	\item  \textbf{Sudionici:} baza podataka
	\item  \textbf{Preduvjet:} -
	\item  \textbf{Opis osnovnog tijeka:}
	
	\item[] \begin{packed_enum}
		
		\item korisnik odabire opciju za registraciju
		\item korisnik unosi potrebne korisničke podatke
		\item korsinik prima obavijest o uspješnoj registraciji
	\end{packed_enum}
	
	\item  \textbf{Opis mogućih odstupanja:}
	
	\item[] \begin{packed_item}
		
		\item[2.a] Odabir vec zauzetog korisničkog imena i/ili e-maila, unos korisničkog podatka u nedozvoljenom formatu ili unos neispravnoga e-maila 
		\item[] \begin{packed_enum}
			
			\item sustav obaviještava korisnika o neispravnom unosu i ponovno ga vraća na stranicu za registraciju
			\item korisnik mijenja potrebne podatke i završava unos ili odustaje od registracije
			
		\end{packed_enum}
	\end{packed_item}
\end{packed_item}



\subsubsection{Dijagrami obrazaca uporabe}

\textit{Prikazati odnos aktora i obrazaca uporabe odgovarajućim UML dijagramom. Nije nužno nacrtati sve na jednom dijagramu. Modelirati po razinama apstrakcije i skupovima srodnih funkcionalnosti.}
\eject		

\subsection{Sekvencijski dijagrami}

\textbf{\textit{dio 1. revizije}}\\

\textit{Nacrtati sekvencijske dijagrame koji modeliraju najvažnije dijelove sustava (max. 4 dijagrama). Ukoliko postoji nedoumica oko odabira, razjasniti s asistentom. Uz svaki dijagram napisati detaljni opis dijagrama.}
\eject

\section{Ostali zahtjevi}

\textbf{\textit{dio 1. revizije}}\\

\textit{Nefunkcionalni zahtjevi i zahtjevi domene primjene dopunjuju funkcionalne zahtjeve. Oni opisuju \textbf{kako se sustav treba ponašati} i koja \textbf{ograničenja} treba poštivati (performanse, korisničko iskustvo, pouzdanost, standardi kvalitete, sigurnost...). Primjeri takvih zahtjeva u Vašem projektu mogu biti: podržani jezici korisničkog sučelja, vrijeme odziva, najveći mogući podržani broj korisnika, podržane web/mobilne platforme, razina zaštite (protokoli komunikacije, kriptiranje...)... Svaki takav zahtjev potrebno je navesti u jednoj ili dvije rečenice.}
