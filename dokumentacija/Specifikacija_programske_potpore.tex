\chapter{Specifikacija programske potpore}

\section{Funkcionalni zahtjevi}

\textbf{\textit{dio 1. revizije}}\\



\noindent \textbf{Dionici:}

\begin{packed_enum}
	
	\item Razvojni tim
	\item Administrator			
	\item Korsinik
	\item Vanjski suradnici - CROZ
	
\end{packed_enum}

\noindent \textbf{Aktori i njihovi funkcionalni zahtjevi:}


\begin{packed_enum}
	\item  \underbar{Korisnik (inicijator) može:}
	
	\begin{packed_enum}
		
		\item poprimiti ulogu autora zahtjeva 
		\begin{packed_enum}
			
			\item   može zadati zahtjev za pomoć
			\item  može obrisati ili blokirati unešeni zahtjev
			
		\end{packed_enum}
		\item poprimiti ulogu izvršitelja zahtjeva 
		\begin{packed_enum}
			
			\item  može odabrati zahtjev s liste aktivnih zahtjeva
			
		\end{packed_enum}
		
		\item razmjenjivati notifikacije kada izvršitelj odabere zahtjev
		\item dohvatiti profil drugog korisnika u trenutku pregleda zahtjeva
		\item ocjenjivati i komentirati druge korisnike
		\item vidjeti "lanac povjerenja"
		\item pregledati listu svojih izvršenih i ponuđenih zahtjeva 
	\end{packed_enum}
	
	\item  \underbar{Administrator (inicijator) može:}
	
	\begin{packed_enum}
		
		\item kontrolirati sadržaj koji se objavljuje
		\item brisanje zahtjeva
		\item blokiranje korisnika pristupu aplikaciji 
	\end{packed_enum}
	
	\item  \underbar{Baza podataka (sudionik) može:}
	\begin{packed_enum}
		
		\item spremanje podataka o profilima, zahtjevima, razmjeni notifikacija
		
	\end{packed_enum}
	
	\item  \underbar{Poslužitelj (sudionik) može:}
	\begin{packed_enum}
		
		\item spremanje podataka o profilima, zahtjevima, razmjeni notifikacija
		
	\end{packed_enum}
\end{packed_enum}

\eject 



\subsection{Obrasci uporabe}

\textbf{\textit{dio 1. revizije}}

\subsubsection{Opis obrazaca uporabe}
\textit{Funkcionalne zahtjeve razraditi u obliku obrazaca uporabe. Svaki obrazac je potrebno razraditi prema donjem predlošku. Ukoliko u nekom koraku može doći do odstupanja, potrebno je to odstupanje opisati i po mogućnosti ponuditi rješenje kojim bi se tijek obrasca vratio na osnovni tijek.}\\


\noindent \underbar{\textbf{UC1 - Registracija}}
\begin{packed_item}
	\item \textbf{Glavni sudionik: } Javni korisnik
	\item  \textbf{Cilj:} stvoriti korisnički račun za pristup sustavu
	\item  \textbf{Sudionici:} baza podataka
	\item  \textbf{Preduvjet:} -
	\item  \textbf{Opis osnovnog tijeka:}
	
	\item[] \begin{packed_enum}
		
		\item korisnik odabire opciju za registraciju
		\item korisnik unosi potrebne korisničke podatke
		\item korsinik prima obavijest o uspješnoj registraciji
	\end{packed_enum}
	
	\item  \textbf{Opis mogućih odstupanja:}
	
	\item[] \begin{packed_item}
		
		\item[2.a] Odabir vec zauzetog korisničkog imena i/ili e-maila, unos korisničkog podatka u nedozvoljenom formatu ili unos neispravnoga e-maila 
		\item[] \begin{packed_enum}
			
			\item sustav obaviještava korisnika o neispravnom unosu i ponovno ga vraća na stranicu za registraciju
			\item korisnik mijenja potrebne podatke i završava unos ili odustaje od registracije
			
		\end{packed_enum}
	\end{packed_item}
\end{packed_item}

\noindent \underbar{\textbf{UC2 -Prijava}}
\begin{packed_item}
	\item \textbf{Glavni sudionik: }Registrirani korisnik
	\item  \textbf{Cilj:} dobiti pristup korisničkom sučelju
	\item  \textbf{Sudionici:} baza podataka
	\item  \textbf{Preduvjet:} registracija
	\item  \textbf{Opis osnovnog tijeka:}
	
	\item[] \begin{packed_enum}
		
		\item unos korisničkog imena i lozinke
		\item potvrda o ispravnosti unesenih podataka
		\item pristup korisničkim funkcijama
	\end{packed_enum}
	
	\item  \textbf{Opis mogućih odstupanja:}
	
	\item[] \begin{packed_item}
		
		\item[1.a] neispravan unos korisničkog imena ili lozinke
		\item[] \begin{packed_enum}
			
			\item sustav obaviještava korisnika o neispravnom unosu i ponovno ga vraća na stranicu za prijavu
			
		\end{packed_enum}
	\end{packed_item}
\end{packed_item}

\noindent \underbar{\textbf{UC3 -Zadavanje zahtjeva za pomoć}}
\begin{packed_item}
	\item \textbf{Glavni sudionik: } Registrirani korisnik
	\item  \textbf{Cilj:} kreirati novi zahtjev za pomoć
	\item  \textbf{Sudionici:} baza podataka
	\item  \textbf{Preduvjet:} korisnik je prijavljen
	\item  \textbf{Opis osnovnog tijeka:}
	
	\item[] \begin{packed_enum}
		
		\item korisnik odabire opciju da traži pomoć
		\item pojavljuje se obrazac za kreiranje zahtjeva za pomoći
		\item korisnik odabire opciju "Objavi zahtjev"
	\end{packed_enum}
\end{packed_item}
\noindent \underbar{\textbf{UC3.1 - Određivanje lokacije uređaja}}
\begin{packed_item}
	\item \textbf{Glavni sudionik: } Servis za lociranje uređaja
	\item  \textbf{Cilj:} dohvatiti trenutnu lokaciju korisnika
	\item  \textbf{Sudionici:} korisnik
	\item  \textbf{Preduvjet:} korisnik je prijavljen, zadaje se zahtjev za pomoć
	\item  \textbf{Opis osnovnog tijeka:}
	
	\item[] \begin{packed_enum}
		
		\item korisnik omogućava aplikaciji pristup vlastitoj lokaciji
		\item korisnik potvrđuje očitanu lokaciju
		\item očitana lokacija se sprema u bazu
	\end{packed_enum}
	\item  \textbf{Opis mogućih odstupanja:}
	
	\item[] \begin{packed_item}
		
		\item[1.a] korsnik ne odobrava pristup njegovoj lokaciji, ona nije dostupna ili je krivo očitana
		\item[] \begin{packed_enum}
			
			\item aplikacija nudi opciju postavljanja lokacije pomoću karte
			
		\end{packed_enum}
	\end{packed_item}
\end{packed_item}
\noindent \underbar{\textbf{UC3.2 - Zadavanje lokacije pomoću karte}}
\begin{packed_item}
	
	\item \textbf{Glavni sudionik: } Servis karte
	\item  \textbf{Cilj:} postaviti željenu lokaciju pomoću karte
	\item  \textbf{Sudionici:} registrirani korisnik
	\item  \textbf{Preduvjet:} korisnik je prijavljen, zadaje se zahtjev za pomoć
	\item  \textbf{Opis osnovnog tijeka:}
	
	\item[] \begin{packed_enum}
		
		\item korisnik odabire opciju postavljanja lokacije pomoću karte
		\item servis karte dohvaća kartu te korisnik precizno odabire željeno područje
		\item odabrana lokacija se sprema u bazu
	\end{packed_enum}
\end{packed_item}
\noindent \underbar{\textbf{UC4 - Pregled liste korisnika}}
\begin{packed_item}
	
	\item \textbf{Glavni sudionik: }Registrirani korisnik, administrator
	\item  \textbf{Cilj:} Pregledati listu registriranih korisnika
	\item  \textbf{Sudionici:} Baza podataka
	\item  \textbf{Preduvjet:} Korisnik je prijavljen
	\item  \textbf{Opis osnovnog tijeka:}
	
	\item[] \begin{packed_enum}
		
		\item Korisnik/administrator u bilo kojem trenutku, pritiskom na gumb može pretražiti registrirane korisnike
		\item Iz baze se dohvaća lista profila korisnika
	\end{packed_enum}
\end{packed_item}
\noindent \underbar{\textbf{UC4.1 - Administratorski pregled liste korisnika}}
\begin{packed_item}
	
	\item \textbf{Glavni sudionik: }Administrator
	\item  \textbf{Cilj:} Pregledati listu registriranih korisnika
	\item  \textbf{Sudionici:} Baza podataka, registrirani korisnik
	\item  \textbf{Preduvjet:} Korisnik je prijavljen kao administrator
	\item  \textbf{Opis osnovnog tijeka:}
	
	\item[] \begin{packed_enum}
		
		\item Administrator u bilo kojem trenutku, pritiskom na gumb može pretražiti registrirane korisnike
		\item Iz baze se dohvaća lista profila korisnika
	\end{packed_enum}
\end{packed_item}

\noindent \underbar{\textbf{UC4.2 - Korisnički pregled liste korisnika}}
\begin{packed_item}
	
	\item \textbf{Glavni sudionik: }Registrirani korisnik
	\item  \textbf{Cilj:} Pregledati listu registriranih korisnika
	\item  \textbf{Sudionici:} Baza podataka, administrator
	\item  \textbf{Preduvjet:} Korisnik je prijavljen
	\item  \textbf{Opis osnovnog tijeka:}
	
	\item[] \begin{packed_enum}
		
		\item Korisnik u bilo kojem trenutku, pritiskom na gumb može pretražiti registrirane korisnike
		\item Iz baze se dohvaća lista profila korisnika
	\end{packed_enum}
\end{packed_item}

\noindent \underbar{\textbf{UC5 - Prikaz liste aktivnih zahtjeva}}
\begin{packed_item}
	
	\item \textbf{Glavni sudionik: }Registrirani korisnik, administrator
	\item  \textbf{Cilj:} pregledati aktivne zahtjeve i ponuditi pomoć
	\item  \textbf{Sudionici:} Baza podataka
	\item  \textbf{Preduvjet:} Korisnik je prijavljen
	\item  \textbf{Opis osnovnog tijeka:}
	
	\item[] \begin{packed_enum}
		
		\item Korisnik odabire ulogu izvršitelja zahtjeva
		\item Iz baze se dohvaća lista aktivnih zahtjeva autora koji trebaju pomoć te se nalaze unutar jednog kilometra od lokacije izvršitelja
	\end{packed_enum}
	
	\item  \textbf{Opis mogućih odstupanja:}
	
	\item[] \begin{packed_item}
		
		\item[2.a] Ne postoje aktivni zahtjevi u označenom području
		\item[] \begin{packed_enum}
			
			\item Otvara se mogućnost proširenja liste na veće geografsko područje
			
		\end{packed_enum}
	\end{packed_item}
\end{packed_item}
\noindent \underbar{\textbf{UC5.1 - Administratorski prikaz liste aktivnih zahtjeva}}
\begin{packed_item}
	
	\item \textbf{Glavni sudionik: }Administrator
	\item  \textbf{Cilj:} Pregledati aktivne zahtjeve
	\item  \textbf{Sudionici:} Baza podataka
	\item  \textbf{Preduvjet:} Korisnik je prijavljen kao administrator
	\item  \textbf{Opis osnovnog tijeka:}
	
	\item[] \begin{packed_enum}
		
		\item Administrator odabire opciju pregleda liste aktivnih zahtjeva
		\item Iz baze se dohvaća lista aktivnih zahtjeva autora koji trebaju pomoć te se nalaze unutar jednog kilometra od lokacije administratora
	\end{packed_enum}
	
	\item  \textbf{Opis mogućih odstupanja:}
	
	\item[] \begin{packed_item}
		
		\item[2.a] Ne postoje aktivni zahtjevi u označenom području
		\item[] \begin{packed_enum}
			
			\item Otvara se mogućnost proširenja liste na veće geografsko područje
			
		\end{packed_enum}
	\end{packed_item}
\end{packed_item}
\noindent \underbar{\textbf{UC5.2 - Korisnički prikaz liste aktivnih zahtjeva}}
\begin{packed_item}
	
	\item \textbf{Glavni sudionik: }Registrirani korisnik
	\item  \textbf{Cilj:} Pregledati aktivne zahtjeve i ponuditi pomoć
	\item  \textbf{Sudionici:} Baza podataka
	\item  \textbf{Preduvjet:} Korisnik je prijavljen
	\item  \textbf{Opis osnovnog tijeka:}
	
	\item[] \begin{packed_enum}
		
		\item Korisnik odabire ulogu izvršitelja zahtjeva
		\item Korisnik odabire opciju pregleda aktivnih zahtjeva
		\item Iz baze se dohvaća lista aktivnih zahtjeva autora koji trebaju pomoć te se nalaze unutar jednog kilometra od lokacije izvršitelja
	\end{packed_enum}
	
	\item  \textbf{Opis mogućih odstupanja:}
	
	\item[] \begin{packed_item}
		
		\item[2.a] Ne postoje aktivni zahtjevi u označenom području
		\item[] \begin{packed_enum}
			
			\item Otvara se mogućnost proširenja liste na veće geografsko područje
			
		\end{packed_enum}
	\end{packed_item}
\end{packed_item}

\noindent \underbar{\textbf{UC6 - Dodjela administrativnog područja}}
\begin{packed_item}
	\item \textbf{Glavni sudionik: }Administrator
	\item  \textbf{Cilj:} dodjela administrativnog područja prema geografskoj lokaciji
	\item  \textbf{Sudionici:} Sustav
	\item  \textbf{Preduvjet:} Korisnik je registriran i prijavljen kao administrator
	\item  \textbf{Opis osnovnog tijeka:}
	
	\item[] \begin{packed_enum}
		
		\item Administratoru preko vlastite lokacije sustav dodjeljuje administrativno područje
	\end{packed_enum}
\end{packed_item}

\noindent \underbar{\textbf{UC9 - Pregled profila}}
\begin{packed_item}
	
	\item \textbf{Glavni sudionik: } Registrirani korisnik
	\item  \textbf{Cilj:} Dobiti pristup korisničkom profilu
	\item  \textbf{Sudionici:} Baza podataka
	\item  \textbf{Preduvjet:} Korisnik je prijavljen, prikaz liste aktivnih zahtjeva, pregled liste korisnika
	\item  \textbf{Opis osnovnog tijeka:}
	
	\item[] \begin{packed_enum}
		
		\item korisnik odabire opciju pregleda profila
		\item Iz baze se dohvaća profil i prikazuje se korisniku
	\end{packed_enum}
\end{packed_item}
\noindent \underbar{\textbf{UC9.1 - Pregled vlastitog profila i administratorski pregled}}
\begin{packed_item}
	
	\item \textbf{Glavni sudionik: } Registrirani korisnik
	\item  \textbf{Cilj:} Pregledati vlastiti profil
	\item  \textbf{Sudionici:} Baza podataka, administrator
	\item  \textbf{Preduvjet:} Korisnik je prijavljen
	\item  \textbf{Opis osnovnog tijeka:}
	
	\item[] \begin{packed_enum}
		
		\item Korisnik odabire opciju pregleda vlastitog profila
		\item Iz baze se dohvaća korisnikov profil i vlastiti podaci
	\end{packed_enum}
\end{packed_item}
\noindent \underbar{\textbf{UC9.2 - Pregled profila drugog korisnika}}
\begin{packed_item}
	
	\item \textbf{Glavni sudionik: } Registrirani korisnik
	\item  \textbf{Cilj:} Dobiti pristup tuđim korisničkim profilima
	\item  \textbf{Sudionici:} Baza podataka
	\item  \textbf{Preduvjet:} Korisnik je prijavljen, prikaz liste aktivnih zahtjeva, pregled liste korisnika
	\item  \textbf{Opis osnovnog tijeka:}
	
	\item[] \begin{packed_enum}
		
		\item Otvara se opcija pregleda profila autora zahtjeva
		\item iz baze se dohvaća profil o autoru zahtjeva kao i prikaz "lanca povjerenja"
	\end{packed_enum}
	\item  \textbf{Opis mogućih odstupanja:}
	
	\item[] \begin{packed_item}
		
		\item[1.a] Korisnik čiji profil želimo pregledati je blokiran od strane administratora
		\item[] \begin{packed_enum}
			
			\item Sustav vraća korisnika na početnu stranicu
			
		\end{packed_enum}
		
	\end{packed_item}
\end{packed_item}

\noindent \underbar{\textbf{UC11 - Blokiranje korisnika}}
\begin{packed_item}
	\item \textbf{Glavni sudionik: }Administrator
	\item  \textbf{Cilj:} Mogućnost blokiranja korisnika
	\item  \textbf{Sudionici:} -
	\item  \textbf{Preduvjet:} Korisnik je registriran i prijavljen kao administrator, administratorski pregled liste korisnika
	\item  \textbf{Opis osnovnog tijeka:}
	
	\item[] \begin{packed_enum}
		
		\item Pri pregledu liste korisnika, administratoru se omogućuje blokiranje korisničkih računa
	\end{packed_enum}
\end{packed_item}


\subsubsection{Dijagrami obrazaca uporabe}

\textit{Prikazati odnos aktora i obrazaca uporabe odgovarajućim UML dijagramom. Nije nužno nacrtati sve na jednom dijagramu. Modelirati po razinama apstrakcije i skupovima srodnih funkcionalnosti.}
\eject		

\subsection{Sekvencijski dijagrami}

\textbf{\textit{dio 1. revizije}}\\

\textit{Nacrtati sekvencijske dijagrame koji modeliraju najvažnije dijelove sustava (max. 4 dijagrama). Ukoliko postoji nedoumica oko odabira, razjasniti s asistentom. Uz svaki dijagram napisati detaljni opis dijagrama.}
\eject

\section{Ostali zahtjevi}

\textbf{\textit{dio 1. revizije}}\\

\textit{Nefunkcionalni zahtjevi i zahtjevi domene primjene dopunjuju funkcionalne zahtjeve. Oni opisuju \textbf{kako se sustav treba ponašati} i koja \textbf{ograničenja} treba poštivati (performanse, korisničko iskustvo, pouzdanost, standardi kvalitete, sigurnost...). Primjeri takvih zahtjeva u Vašem projektu mogu biti: podržani jezici korisničkog sučelja, vrijeme odziva, najveći mogući podržani broj korisnika, podržane web/mobilne platforme, razina zaštite (protokoli komunikacije, kriptiranje...)... Svaki takav zahtjev potrebno je navesti u jednoj ili dvije rečenice.}
